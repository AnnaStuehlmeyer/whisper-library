\documentclass[colorback,accentcolor=tud2a,12pt,paper=a4]{tudreport}

\usepackage{ngerman}
\usepackage[T1]{fontenc}
\usepackage[latin1]{inputenc}
\usepackage{helvet}
\usepackage{parcolumns}

\newcommand{\titlerow}[2]{
	\begin{parcolumns}[colwidths={1=.15\linewidth}]{2}
		\colchunk[1]{#1:} 
		\colchunk[2]{#2}
	\end{parcolumns}
	\vspace{0.2cm}
}

\title{Instant Message Whispering via Covert Channels}
\subtitle{Qualit�tssicherungsdokument}
\subsubtitle{
	\titlerow{Gruppe NR}{
		Jan Simon Bunten <mail@xyz.de>\\
		Simon Kadel <mail@xyz.de>\\
		Martin Sven Oehler <mail@xyz.de>\\
		Arne Sven St�hlmeier <mail@xyz.de>}
	\titlerow{Teamleiter}{Philipp Pl�hn <mail@xyz.de>}
	\titlerow{Auftraggeber}{
		Titel Carlos Garcia <carlos.garcia@cased.de>\\
		FG Telekooperation\\
		FB 20 - Informatik}
	\titlerow{Abgabedatum}{15.2.2014}}
\institution{Bachelor-Praktikum WS 2013/2014\\Fachbereich Informatik}

\begin{document}

	\maketitle
	\tableofcontents 
	
	\chapter{Einleitung}
		Kurze Projektbeschreibung
	
	\chapter{Qualit�sziele}

	\section{Zuverl�ssigkeit}
		Der Benutzer eine Bibliothek verl�sst sich darauf, dass sie korrekt funktioniert und das tut, was in der Dokumentation festgehalten ist. Deshalb ist Zuverl�ssigkeit ist f�r eine Bibliothek unbedingt notwendig.
		
		Die Zuverl�ssigkeit kann durch Testen verbessert werden. Deshalb benutzen wir die boost.test Bibliothek, die automatische Tests in C++ erm�glicht. F�r jede Methode wird mindestens ein Test geschrieben und f�r jede Aufgabe (siehe Testbarkeit) mindestens 2 Tests. Diese werden mindestens einmal pro Woche auf der aktuellen Version der Software komplett ausgef�hrt. Fehler werden im Ticketsystem unseres SCM- Servers eingetragen.
		
		Au�erdem f�hren wir Code Reviews durch. Nach Abschluss eines Use Cases wird der Code von einem an diesem Use Case unbeteiligten Teammitglied mit Hilfe des Tools ... �berpr�ft. M�gliche Fehler werden schnellstm�glich von den Entwicklern behoben. Dann wird der Vorgang wiederholt, bis die Kriterien erf�llt sind.
		
        \section{Testbarkeit}
    		Aus der Zuverl�ssigkeit ergibt sich ein weiters Qualit�tsmerkmal. Wenn man die Software Testen will, muss sie auch testbar sein. Dabei geht es nicht nur um Unittests, sondern vorallem um Integrations- und Systemtests. Damit es einfach ist, diese durchzuf�hren, wird eine klare Beschreibung und Trennung der Aufgaben ben�tigt (Seperations of Concern). Das erreichen wir, indem wir w�hrend der Entwicklung alle Aufgaben unserer Software festhalten und einem unserer Module zuweisen. 
    		
    		Es ist sehr schwierig, Testbarkeit durch Werkzeuge sicher zu stellen. Eine gute Testbarkeit ergibt sich durch eine Architektur der Sofware, dass dieses Qualit�tsmerkmal mit bedenkt.Es muss beim Entwurf mit bedacht werden und w�hrend der Entwicklung stets �berpr�ft werden.
    		
    		Die einfachste M�glichkeit die Testbarkeit zu �berpr�fen ist es, Tests zu schreiben und dabei festzustellen, wie gut das m�glich ist. Wenn dabei auff�llt, dass manche Funktionalit�ten nur schwer oder garnicht zu testen sind, werden wir dies in unseren Teamtreffen besprechen und eine L�sung durch Anpassen der Architektur ausarbeiten.
	        
	
\appendix	
	\chapter{Anhang}
		(Am Ende des Projekts nachzureichen)\\
		Beleg f�r durchgef�hrte Ma�nahmen, bzw. falls nicht durchgef�hrt eine Begr�ndung wieso die Durchf�hrung nicht m�glich oder nicht erfolgt ist. \\
		Weitere Anforderungen sind den Unterlagen und der Vorlesung zur Projektbegleitung zu entnehmen.
	
\end{document}
\documentclass[colorback,accentcolor=tud2a,12pt,paper=a4]{tudreport}

\usepackage{ngerman}
\usepackage[T1]{fontenc}
\usepackage[latin1]{inputenc}
\usepackage{helvet}
\usepackage{parcolumns}

\newcommand{\titlerow}[2]{
	\begin{parcolumns}[colwidths={1=.15\linewidth}]{2}
		\colchunk[1]{#1:} 
		\colchunk[2]{#2}
	\end{parcolumns}
	\vspace{0.2cm}
}

\title{Instant Message Whispering via Covert Channels}
\subtitle{Qualit�tssicherungsdokument}
\subsubtitle{
	\titlerow{Gruppe NR}{
		Jan Simon Bunten <mail@xyz.de>\\
		Simon Kadel <mail@xyz.de>\\
		Martin Sven Oehler <mail@xyz.de>\\
		Arne Sven St�hlmeier <mail@xyz.de>}
	\titlerow{Teamleiter}{Philipp Pl�hn <mail@xyz.de>}
	\titlerow{Auftraggeber}{
		Titel Carlos Garcia <carlos.garcia@cased.de>\\
		FG Telekooperation\\
		FB 20 - Informatik}
	\titlerow{Abgabedatum}{15.2.2014}}
\institution{Bachelor-Praktikum WS 2013/2014\\Fachbereich Informatik}

\begin{document}

	\maketitle
	\tableofcontents 
	
	\chapter{Einleitung}
		Kurze Projektbeschreibung
	
	\chapter{Qualit�sziele}
	\section{Modularit�t} ist kein auf den Folien genanntes QS-Ziel. Stattdessen vllt. Benutzbarkeit? Ma�nahme w�re dann u.a. Modularit�t, aber auch noch mehr.
	\section{Verl�sslichkeit}
        \section{Testbarkeit}
    		Aus derl Verl�sslichkeit ergibt sich ein weiters Qualit�tsmerkmal. Wenn man die Software Testen will, muss sie auch testbar sein. Dabei geht es nicht nur um Unittests, sondern vorallem um Integrations- und Systemtests. Damit es einfach ist, diese durchzuf�hren, wird eine klare Beschreibung und Trennung der Aufgaben ben�tigt (Seperations of Concern). Das erreichen wir, indem wir w�hrend der Entwicklung alle Aufgaben unserer Software festhalten und einem unserer Module zuweisen. 
    		
    		Es ist sehr schwierig, Testbarkeit durch irgendwelche Werkzeuge sicher zu stellen. Eine gute Testbarkeit ergibt sich durch eine Architektur der Sofware, dass dieses Qualit�tsmerkmal mit bedenkt. �hnlich wie die Modularit�t muss es beim Entwurf mit bedacht werden und w�hrend der Entwicklung stets �berpr�ft werden.
    		
    		Die einfachste M�glichkeit die Testbarkeit zu �berpr�fen ist es, Tests zu schreiben und dabei festzustellen, wie gut das m�glich ist. Wir werden w�hrend der Entwicklung fr�hzeitig anfangen, System- und Integrationstests zu schreiben und zu automatisieren. Wenn dabei auff�llt, dass manche Funktionalit�ten nur schwer oder garnicht zu testen sind, werden wir dies in unseren Teamtreffen besprechen und eine L�sung durch Anpassen der Architektur ausarbeiten.
	        
	
\appendix	
	\chapter{Anhang}
		(Am Ende des Projekts nachzureichen)\\
		Beleg f�r durchgef�hrte Ma�nahmen, bzw. falls nicht durchgef�hrt eine Begr�ndung wieso die Durchf�hrung nicht m�glich oder nicht erfolgt ist. \\
		Weitere Anforderungen sind den Unterlagen und der Vorlesung zur Projektbegleitung zu entnehmen.
	
\end{document}